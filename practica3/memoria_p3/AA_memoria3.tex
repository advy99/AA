\documentclass[12pt, spanish]{article}
\usepackage[spanish]{babel}
\selectlanguage{spanish}
%\usepackage{natbib}
\usepackage{url}
\usepackage[utf8x]{inputenc}
\usepackage{graphicx}
\graphicspath{{images/}}
\usepackage{parskip}
\usepackage{fancyhdr}
\usepackage{vmargin}
\usepackage{multirow}
\usepackage{float}
\usepackage{chngpage}

\usepackage{amsfonts}

\usepackage{subcaption}

\usepackage{hyperref}
\usepackage[
    type={CC},
    modifier={by-nc-sa},
    version={4.0},
]{doclicense}

\hypersetup{
    colorlinks=true,
    linkcolor=blue,
    filecolor=magenta,      
    urlcolor=cyan,
}

% para codigo
\usepackage{listings}
\usepackage{xcolor}



%% configuración de listings

\definecolor{listing-background}{HTML}{F7F7F7}
\definecolor{listing-rule}{HTML}{B3B2B3}
\definecolor{listing-numbers}{HTML}{B3B2B3}
\definecolor{listing-text-color}{HTML}{000000}
\definecolor{listing-keyword}{HTML}{435489}
\definecolor{listing-identifier}{HTML}{435489}
\definecolor{listing-string}{HTML}{00999A}
\definecolor{listing-comment}{HTML}{8E8E8E}
\definecolor{listing-javadoc-comment}{HTML}{006CA9}

\lstdefinestyle{eisvogel_listing_style}{
  language         = python,
%$if(listings-disable-line-numbers)$
%  xleftmargin      = 0.6em,
%  framexleftmargin = 0.4em,
%$else$
  numbers          = left,
  xleftmargin      = 0em,
 framexleftmargin = 0em,
%$endif$
  backgroundcolor  = \color{listing-background},
  basicstyle       = \color{listing-text-color}\small\ttfamily{}\linespread{1.15}, % print whole listing small
  breaklines       = true,
  frame            = single,
  framesep         = 0.19em,
  rulecolor        = \color{listing-rule},
  frameround       = ffff,
  tabsize          = 4,
  numberstyle      = \color{listing-numbers},
  aboveskip        = 1.0em,
  belowskip        = 0.1em,
  abovecaptionskip = 0em,
  belowcaptionskip = 1.0em,
  keywordstyle     = \color{listing-keyword}\bfseries,
  classoffset      = 0,
  sensitive        = true,
  identifierstyle  = \color{listing-identifier},
  commentstyle     = \color{listing-comment},
  morecomment      = [s][\color{listing-javadoc-comment}]{/**}{*/},
  stringstyle      = \color{listing-string},
  showstringspaces = false,
  escapeinside     = {/*@}{@*/}, % Allow LaTeX inside these special comments
  literate         =
  {á}{{\'a}}1 {é}{{\'e}}1 {í}{{\'i}}1 {ó}{{\'o}}1 {ú}{{\'u}}1
  {Á}{{\'A}}1 {É}{{\'E}}1 {Í}{{\'I}}1 {Ó}{{\'O}}1 {Ú}{{\'U}}1
  {à}{{\`a}}1 {è}{{\'e}}1 {ì}{{\`i}}1 {ò}{{\`o}}1 {ù}{{\`u}}1
  {À}{{\`A}}1 {È}{{\'E}}1 {Ì}{{\`I}}1 {Ò}{{\`O}}1 {Ù}{{\`U}}1
  {ä}{{\"a}}1 {ë}{{\"e}}1 {ï}{{\"i}}1 {ö}{{\"o}}1 {ü}{{\"u}}1
  {Ä}{{\"A}}1 {Ë}{{\"E}}1 {Ï}{{\"I}}1 {Ö}{{\"O}}1 {Ü}{{\"U}}1
  {â}{{\^a}}1 {ê}{{\^e}}1 {î}{{\^i}}1 {ô}{{\^o}}1 {û}{{\^u}}1
  {Â}{{\^A}}1 {Ê}{{\^E}}1 {Î}{{\^I}}1 {Ô}{{\^O}}1 {Û}{{\^U}}1
  {œ}{{\oe}}1 {Œ}{{\OE}}1 {æ}{{\ae}}1 {Æ}{{\AE}}1 {ß}{{\ss}}1
  {ç}{{\c c}}1 {Ç}{{\c C}}1 {ø}{{\o}}1 {å}{{\r a}}1 {Å}{{\r A}}1
  {€}{{\EUR}}1 {£}{{\pounds}}1 {«}{{\guillemotleft}}1
  {»}{{\guillemotright}}1 {ñ}{{\~n}}1 {Ñ}{{\~N}}1 {¿}{{?`}}1
  {…}{{\ldots}}1 {≥}{{>=}}1 {≤}{{<=}}1 {„}{{\glqq}}1 {“}{{\grqq}}1
  {”}{{''}}1
}
\lstset{style=eisvogel_listing_style}


\usepackage[default]{sourcesanspro}

\setmarginsrb{2 cm}{1 cm}{2 cm}{2 cm}{1 cm}{1.5 cm}{1 cm}{1.5 cm}

\title{Práctica 3:\\
Programación  \hspace{0.05cm} }                           
\author{Antonio David Villegas Yeguas}                             
\date{\today}                                           

\renewcommand*\contentsname{hola}

\makeatletter
\let\thetitle\@title
\let\theauthor\@author
\let\thedate\@date
\makeatother

\pagestyle{fancy}
\fancyhf{}
\rhead{\theauthor}
\lhead{\thetitle}
\cfoot{\thepage}

\begin{document}

%%%%%%%%%%%%%%%%%%%%%%%%%%%%%%%%%%%%%%%%%%%%%%%%%%%%%%%%%%%%%%%%%%%%%%%%%%%%%%%%%%%%%%%%%

\begin{titlepage}
    \centering
    \vspace*{0.3 cm}
    \includegraphics[scale = 0.50]{ugr.png}\\[0.7 cm]
    %\textsc{\LARGE Universidad de Granada}\\[2.0 cm]   
    \textsc{\large 3º CSI 2019/20 - Grupo 1}\\[0.5 cm]            
    \textsc{\large Grado en Ingeniería Informática}\\[0.5 cm]              
    \rule{\linewidth}{0.2 mm} \\[0.2 cm]
    { \huge \bfseries \thetitle}\\
    \rule{\linewidth}{0.2 mm} \\[1 cm]
    
    \begin{minipage}{0.4\textwidth}
        \begin{flushleft} \large
            \emph{Autor:}\\
            \theauthor\\ 
			 \emph{DNI:}\\
            77021623-M
            \end{flushleft}
            \end{minipage}~
            \begin{minipage}{0.4\textwidth}
            \begin{flushright} \large
            \emph{Asignatura: \\
            AA}   \\     
            \emph{Correo:}\\
            advy99@correo.ugr.es           
        \end{flushright}
    \end{minipage}\\[0.5cm]
  
    {\large \thedate}\\[0.5cm]
    %{\url{https://github.com/advy99/AA/}}
    {\doclicenseThis}
 	
    \vfill
    
\end{titlepage}

%%%%%%%%%%%%%%%%%%%%%%%%%%%%%%%%%%%%%%%%%%%%%%%%%%%%%%%%%%%%%%%%%%%%%%%%%%%%%%%%%%%%%%%%%

\tableofcontents
\pagebreak

%%%%%%%%%%%%%%%%%%%%%%%%%%%%%%%%%%%%%%%%%%%%%%%%%%%%%%%%%%%%%%%%%%%%%%%%%%%%%%%%%%%%%%%%%

\section{Problema de clasificación.}

\subsection{Comprender el problema. Identificar X, Y y F en el problema.}



En este problema tenemos datos sobre imágenes de números escritos a mano por un total de 43 personas, los números escritos por 30 de estas son el conjunto de training dado y los números de las 13 personas restantes conforman el conjunto de test.

Cada número se representa como una imagen de 32 por 32 bitmaps, donde 1 representa que en dicha posición se ha escrito y un 0 no.

\subsubsection{X del problema.}


\subsubsection{Y del problema.}

\subsubsection{F del problema.}

\subsection{Clases de funciones a usar.}

\subsection{Conjuntos de training y test.}

\subsection{Preprocesado de los datos.}

\subsection{Fijar la métrica. Idoneidad sobre el problema.}

\subsection{Técnica de ajuste elegida.}

\subsection{Necesidad de regulación.}

\subsection{Modelos a usar.}

\subsection{Hiperparametros y selección del mejor modelo.}

\subsection{Estimación del error fuera de la muestra usando validación cruzada y comparación con el error en test.}

\subsection{Modelo propuesto y estimación del error fuera de la muestra de este modelo.}




\newpage

\section{Problema de regresión.}

\subsection{Comprender el problema. Identificar X, Y y F en el problema.}


\subsubsection{X del problema.}


\subsubsection{Y del problema.}

\subsubsection{F del problema.}

\subsection{Clases de funciones a usar.}

\subsection{Conjuntos de training y test.}

\subsection{Preprocesado de los datos.}

\subsection{Fijar la métrica. Idoneidad sobre el problema.}

\subsection{Técnica de ajuste elegida.}

\subsection{Necesidad de regulación.}

\subsection{Modelos a usar.}

\subsection{Hiperparámetros y selección del mejor modelo.}

\subsection{Estimación del error fuera de la muestra usando validación cruzada y comparación con el error en test.}

\subsection{Modelo propuesto y estimación del error fuera de la muestra de este modelo.}


\newpage


\section{Referencias, material y documentación usada}


\begin{thebibliography}{9}

\bibitem{teoria}
Diapositivas de teoría

\bibitem{libro}
Learning From Data by Yaser S. Abu-Mostafa, Malik Magdon-Ismail, Hsuan-Tien Lin

\bibitem{documentacion-numpy}
Documentación de NumPy:

\url{https://docs.scipy.org/doc/numpy-1.15.0/reference/generated/numpy.random.choice.html}

\url{https://docs.scipy.org/doc/numpy/reference/generated/numpy.transpose.html}

\url{https://docs.scipy.org/doc/numpy/reference/generated/numpy.dot.html}

\url{https://docs.scipy.org/doc/numpy/reference/generated/numpy.linalg.inv.html}

\url{https://docs.scipy.org/doc/numpy/reference/generated/numpy.c_.html}

\url{https://docs.scipy.org/doc/numpy/reference/generated/numpy.square.html}

\url{https://docs.scipy.org/doc/numpy/reference/generated/numpy.mean.html}

\bibitem{mlr_digitos}
Machine Learning Repository: Optical Recognition of Handwritten Digits Data Set 

\url{https://archive.ics.uci.edu/ml/datasets/optical+recognition+of+handwritten+digits}


\bibitem{mlr_crimen}
Machine Learning Repository: Communities and Crime Data Set  

\url{http://archive.ics.uci.edu/ml/datasets/Communities+and+Crime}

\bibitem{sklearn-linearmodel}
\url{https://scikit-learn.org/stable/modules/linear_model.html}

\bibitem{sklearn-model-selection}
https://scikit-learn.org/stable/modules/classes.html#module-sklearn.model_selection

\end{thebibliography}

\end{document}
